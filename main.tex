\documentclass[11pt, a4paper, german]{article}

\usepackage[english]{babel}
\usepackage[utf8x]{inputenc}
\usepackage{amsmath, amsthm, amssymb}
\usepackage{mathtools} %for \coloneqq
\usepackage{faktor}
\usepackage{tikz-cd} 
%bibliography
\usepackage[noadjust]{cite}
\usepackage{hyperref}
\usepackage{enumitem}
%font
%\usepackage[euler-digits, euler-hat-accent]{eulervm}
%\usepackage{newpxtext}

\theoremstyle{plain}
\newtheorem{theorem}{Theorem}[section]
\newtheorem{corollary}[theorem]{Corollary}
\newtheorem{lemma}[theorem]{Lemma}

\theoremstyle{definition}
\newtheorem{definition}[theorem]{Definiton}
\newtheorem{remark}[theorem]{Remark}
\newtheorem{example}[theorem]{Example}

%frontpage
\usepackage{BA_Titelseite}

\author{Maximilian Rath}
\geburtsdatum{2. Januar 1995}
\geburtsort{Reutlingen}

\date{2.2.2016}
\betreuer{Prof. Dr. Daniel Huybrechts}
\zweitgutachter{Martin Ulirsch}
\institut{Mathematisches Institut}


\title{Motivic Zeta Function}
\ausarbeitungstyp{Bachelorarbeit Mathematik}

%I don't want to write the Grothendieck-Ring over and over
\newcommand{\gring}[1][k]{K_0[\mathcal{V}_#1]}

\DeclareMathOperator{\Spec}{Spec}
\DeclareMathOperator{\Sym}{Sym}
\DeclareMathOperator{\Proj}{Proj}
\DeclareMathOperator{\sgn}{sgn}
\DeclareMathOperator{\chara}{char}
\begin{document}

\maketitle
\tableofcontents
\section{Introduction}

In 1949 Andr\'e Weil, in the paper \cite{weil1949}, formulated his famous conjectures
which predicted certain deep properties of the Hasse--Weil zeta function $Z(X,t)$ for a variety $X$ over a finite field $k = \mathbb{F}_q$.
The zeta function is defined as 
\[
    Z(X,t) = \exp \left(\sum_{m=0}^\infty \frac{|X(\mathbb{F}_{q^m})|}{m} t^m \ \right) \in \mathbb{Q}[[t]].
\]

Dwork first established the rationality of the zeta function in \cite{Dwork}, the most basic of the three conjectures. 
Later, using his new scheme theoretic approach to algebraic geometry, Grothendieck gave another proof for the special case of 
smooth projective varieties (\cite{GroRat}).

%\begin{theorem}
%    Let $X$ be a non-singular projective variety over a finite field $k = \mathbb{F}_q$. Then Z(X,t) is a rational function of t.
%\end{theorem}

In \cite{kapranov}, Kapranov generalized the Hasse--Weil zeta function, defining it not only for varieties over finite fields
but varieties over arbitrary ground fields. 
This \emph{motivic zeta function} has coefficents in the so called \emph{Grothendieck ring of varieties} $\gring[k]$ 
over some fixed field $k$, 
which is the free abelian group generated by isomorphism classes of varieties subject to additional "cut-and-paste" relations
(see Definition \ref{gringdef}), and is given as 
\[
    \zeta_{\mu}(X,t) \coloneqq \sum_{n=0}^{\infty} \mu \left( \left[ \Sym^n(X) \right] \right) t^n
\]
where $\mu$ is a ring homomorphism from $\gring[k]$ to some fixed ring $A$ (also called a \emph{motivic measure}), 
and $Sym^n(X)$ is the $n$-fold symmetric product of $X$ (see Definition \ref{symdef}). 
It is a natural question to ask if this zeta function is still a rational function (at least if $\mu$ takes values in a field).
Kapranov proved this for smooth projective curves (\cite[Thm 1.1.9]{kapranov}) and conjectured that it is true in general.
%TODO: really start a sentence with BUT?
But it turns out that in the motivic setting the zeta function is not as well behaved in terms of rationality as in the classical case:
Larsen and Lunts showed in \cite{MR1996804} that the zeta function fails to be rational for complex surfaces.

\begin{theorem}[{\cite[Thm. 1.6]{MR1996804}}]
\label{irrational}
Assume that $k = \mathbb{C}$. There exists a field $\mathcal{H}$ and a motivic measure $\mu_h \colon \gring \to \mathcal{H}$ with the following
property: if $X$ is a smooth complex projective surface such that the geometric genus $P_g(X)=h^{2,0}(X)$ is at least 2, 
then the zeta-function $\zeta_{\mu_h}(X,t)$ is not a rational function.
\end{theorem}

The goal of this thesis is to give a simplified proof of this result, using results about the grothendieck ring that were not available
to Larsen and Lunts.
In this thesis we will first introduce the Grothendieck ring in Section \ref{sec1} and investigate its structure. 
This culminates in a structure theorem due to Bittner (\cite[Thm 3.1]{Bittner}) which will allow us to substantially simplify the original 
proof.
The second section is devoted to the definition of the motivic zeta function, which entails defining and giving examples for the symmetric 
product. There we will also see the connection between the classical Hasse--Weil zeta function and the motivic one.
Construction of the motivic measure $\mu_h$ under wich $Z_{\mu_h}(X,t)$ becomes irrational is content of Section \ref{const} while irrationality
is then finally proven in the last section.

\section{German Summary}
In dieser Bachelorarbeit soll ein Resultat von Larsen und Lunts (\cite{MR1996804}) über die Rationalität der motivischen Zetafunktion
bewiesen werden. Die Motivische Zetafunktion ist eine Verallgemeinerung der Hasse--Weil Zetafunktion für Varietäten über endlichen Körpern
mit Koeffizienten im Grothendieckring der Varietäten $\gring[k]$ der von Isomorphieklassen von algebraischen Varietäten erzeugt wird 
zwischen denen gewissen zusätzliche "cut-and-paste" Relationen bestehen (Siehe \ref{gringdef} für die genaue Definition). 
Diese motivische Zetafunktion ist, im Gegensatz zur Hasse-Weil Zetafunktion, im allgemeinen nicht rational. 
Konkret werden wir folgendes Resultat zeigen.
\begin{theorem}[{\cite[Thm. 1.6]{MR1996804}}]
Sei $k = \mathbb{C}$. Dann existiert ein Körper $\mathcal{H}$ und ein Ringhomomorphismus $\mu_h \colon \gring \to \mathcal{H}$ mit der folgenden
Eigenschaft: Falls $X$ eine glatte, projektive, komplexe Fläche mit geometrischem Genus $P_g(X) = h^{2,0}(X)$ von mindestens 2 ist, dann ist die
motivische Zetafunktion $\zeta_{\mu_h}(X,t)$ nicht rational.
\end{theorem}
Wir werden ein Theorem von Bittner (\cite{Bittner}) verwenden um den ursprünglichen Beweis deutlich kompakter darstellen zu können.



%We now want to motivate what a possible generalization of the Hasse-Weil zeta function for varieties over a arbitrary field $k$ could be.
%This part therefore, in some places, sacrifices mathematical precision for the sake of brevity.  
%We still give citations for most of the results and will of course develop everything that is important for the presentation and proof 
%of our main result in later sections in full detail.
%
%Two things do not generalize easily: the passing to bigger and bigger extensions of the ground field $k$ (what if $k$ is algebraically closed,
%for example?), and counting $k$-rational points (again, if $k = \bar{k}$ there are, in general, a lot of $k$-rational points). 
%
%%XXX this section might be too much to develop in detail
%To see what could be generalizations for these phenomena, 
%we give another equivalent definition of the Hasse--Weil zeta function in terms of effective 0-cycles. 
%Recall that a 0-cycle on a variety $X$ is an element of the free abelian group over the closed 
%points of $X$. Such a cycle $\alpha = \sum_i n_i x_i$ is called \emph{effective} if the $n_i$ are greater than or equal to zero.
%The \emph{degree} of $\alpha$ is given by $\deg(\alpha) = \sum_i n_i \deg(x_i)$ 
%(with $\deg(x_i)$ being the degree of the field extension $\kappa(x_i)/k$).
%
%Using this notation we can rewrite the zeta function as follows 
%\begin{equation}
%    Z(X,t) = \sum_{\alpha} t^{\deg(\alpha)}.
%\end{equation}
%Where the sum ranges over all effective 0-cycles on $X$.
%(For a proof of this (elementary) identity see for example Musta\c{t}\u{a}'s great notes on zeta functions in algebraic geometry 
%\cite[Rem. 2.9]{mustata})
%
%This definition still does not easily generalize, since for example $\mathbb{A}_{\mathbb{C}}^1$ has infinitely many 0-cycles of degree $n$.
%
%But counting degree zero cycles of a given degree $n$, is the same as counting $k$-rational points of the $n$-fold symmetric product of the 
%variety $X$ (denoted by $\Sym^n(X)$).
%We will not formally introduce the symmetric product until later, but thinking of it as the quotient of 
%the $n$-fold product of $X$ with itself by the natural action of the symmetric group in $n$ letters gives the right intuition. 
%For $k$ algebraically closed, the bijection is obvious, since then the $k$-rational points are exactly the closed points. 
%Dividing out by the action of the symmetric group hence introduces the commutativity-relations we have if we consider formal sums
%of such closed points. This intuition carries over for the case of finite fields, hence we get
%
%\begin{equation}
%   Z(X,t) = \sum_{n} |\Sym^n(X)(k)| t^n.
%\end{equation}
%
%So, in some rough sense, we are "measuring" the size of bigger and bigger symmetric products of X, so we expect the symmetric product
%to make an apearance in our generalized zeta function, taking the role of counting solutions over bigger and bigger field-extensions.


\section{Motivic Measures on the Grothendieck Ring of Varieties}
\label{sec1}
In the following the term \emph{$k$-variety} always means a separated, reduced scheme of finite type over a field $k$.
We will write $\mathcal{V}_k$ for the category of $k$-varieties. 

\begin{definition}
    \label{gringdef}
    Let $k$ be a perfect field. 
    Consider the abelian group of formal linear combinations of isomorphism classes of varieties, subject to relations of the form
    \[
        [X \setminus Y] = [X] - [Y]
    \] where $Y$ is closed in $X$.
    With multiplication given by
    \[
        [X][Y] = [X \times Y]
    \]
    this datum forms a ring, called the \emph{Grothendieck ring of varieties} and denoted by $\gring$. 
    
    A \emph{motivic measure} with values in a ring $A$ is a ring homomorphism $\mu \colon \gring \to A$. The identity function
    $\mathrm{id} \colon \gring \to \gring$ is called the \emph{universal motivic measure}.
\end{definition}

To see that this is really a ring we actually have to check that multiplication is compatible with the relations, but we have
\begin{align*}
    [Z] ([X \setminus Y])   &= [Z \times (X \setminus Y)] = [(Z \times X) \setminus (Z \times Y)] \\
                            &= [Z \times X] - [Z \times Y] = [Z]([X] - [Y])
\end{align*} 

\begin{remark}
The reason that we require $k$ to be perfect is that for nonperfect $k$ the product of two varieties need not be reduced. Take for example
$k = k_0(t)$ with $k_0$ of characteristic $p > 0$ and $X = \Spec(k[x]/(x^p - t))$, $Y = \Spec(K) \coloneqq \Spec(k(t^{\frac{1}{p}}))$. 
Then $X$ is a variety since $x^p - t$ is irreducible in $k_0(t)$. The product, on the other hand, can be described as
    \[
        X \times Y = \Spec(K[x]/(x^p t)) = \Spec(K[x]/((x-t^{\frac{1}{p}})^p)
    \]
which is not reduced. We might adapt the definition to this case by changing the defintion of the product to 
$[X][Y] \coloneqq [(X \times Y)_{\text{red}}]$, taking the induced reduced scheme structure on $X \times Y$, but since we are only working
in characteristic zero or over finite fields we will not bother with this technical issue.
\end{remark}

Since the product of varities is commutative up to isomorphism this is a commutative ring with 1 (equal to $[\Spec(k)]$). 
The cut and paste relation furthermore gives us
\[
    0 = [\emptyset] - [\emptyset] = [\emptyset \setminus \emptyset] = [\emptyset]
\]

\begin{example}
    For a finite field $k$, consider the function \footnote{not bothering with any set-theoretic issues} 
    \begin{align*}
        \psi \colon \mathcal{V}_k & \to   \mathbb{Z}\\
        X & \mapsto |X(k)|
    \end{align*}
    Observe that $\psi$ has the following properties:
    \begin{enumerate}
        \item If $X$ and $Y$ are isomorphic $k$-varieties, then $\psi(X) = \psi(Y)$.
        \item If $Y \subset X$ is closed subvariety, then $\psi(X \setminus Y) = \psi(X) - \psi(Y)$.
        \item If $X$ and $Y$ are two varieties, then $\psi(X \times Y) = \psi(X)\psi(Y)$.
    \end{enumerate}
    Hence $\psi$ factors over $\gring[k]$ and gives rise to a motivic measure.
\end{example}

\begin{example}
    \label{projSum}
    Using the decomposition $\mathbb{P}_k^n = \mathbb{P}_k^{n-1} \coprod \mathbb{A}_k^n$ where $\mathbb{P}_k^{n-1}$ is a hyperplane
    in $\mathbb{P}_k^n$  we get $[\mathbb{P}_k^n] = [\mathbb{P}_k^{n-1}] + [\mathbb{A}_k^n]$ in the Grothendieck ring.
    Inductively this yields the identity 
    \[
        [\mathbb{P}_k^n] = \sum_{m=0}^n \mathbb{L}^m
    \]
    %XXX: more natural place for introducing this convention?
    where we write $\mathbb{L}$ for the isomorphism class of the affine line.
\end{example}

This calculation also gives rise to an example of two irreducible varieties that have the same equivalence class in $\gring[k]$ but are not
isomorphic: $\mathbb{P}^n \times (\mathbb{A}^1\setminus \{0\})$ and $\mathbb{A}^{n+1}$. These are not isomorphic since the global section of the
former are $k[x]_{x}$ , but in the Grothendieck ring we get
\begin{align*}
    [\mathbb{P}^n \times \left (\mathbb{A}^1 \setminus \{0\}\right )] &= [\mathbb{P}^n]\left ([\mathbb{A}^1] - 1\right ) \\
                                                                      &= \sum_{m=0}^n \mathbb{L}^{m+1} - \sum_{m=0}^n \mathbb{L}^m \\
                                                                      &= \mathbb{L}^{n+1}
\end{align*}

%XXX: move?

We now investigate the structure of the Grothendieck ring. First we note that it suffices to take irreducible varieties to generate $\gring[k]$.
To see this, take a variety $X$ with irreducible componets $X_1, \dots, X_n$ and set ${U_i \coloneqq X_i \setminus \bigcup_{i \neq j} X_j}$ and
$U \coloneqq \bigcup_i U_i$. By construction, the last union is a disjoint union of closed subsets of $U$, 
and hence $[U] = \sum_i [U_i]$. Together this gives
\[
    [X] = [X \setminus U] + \sum_i [U_i].
\] Now the $U_i$, as open subsets of irreducible sets, are irreducible and $[X \setminus U]$ has dimension
strictly smaller than the dimension of $X$ since we removed a proper open subset from each irreducible component. Hence we can inductively
rewrite $[X \setminus U]$, loosing at least one dimension in each step, with the base case of $\dim  X = 0$ being trivial, as $X$ is then
just a collection of points.

In characteristic zero we can use a weak form of Hironaka's Theorem on the resolution of singularites, namely that for every irreducible, 
projective variety $X$ there is a smooth, projective variety that is birational to $X$ to restrict the set of needed generators even more. 
This observation culminates in a structure theorem about the Grothendieck ring proven by Bittner in \cite{Bittner}.
%TODO: cite hironaka


\begin{theorem}[{\cite[Thm. 3.1]{Bittner}}]
    \label{bittner}
    Let $k$ be a field of characteristic zero. Then $\gring[k]$, as an abelian group, is generated by smooth, integral, complete varieties.
    Furthermore, if $k$ is algebraically closed, it suffices to consider relations of the form
    \[
        [X] - [f^{-1}(Z)] = [Y] - [Z]
    \]
    where $X$ and $Y$ are smooth complete varieties and $f \colon X \to Y$ is a morphism which is a blowup with a smooth center $Z \subset Y$. 
\end{theorem}
\begin{proof}
    We will only proof the first part of the statement, for a proof of the second part see \cite{Bittner}. As remarked before, we can restrict our
    attention to irreducible varietes.  We once again argue by induction on the dimension.
    Let $X$ be an irreducible variety of dimension $n$. By choosing some nonempty affine subvariety and passing to its projective closure we
    find $X'$ projective, irreducible and birational to $X$. Resolving singularities we find $\widetilde{X}$ smooth, projective, irreducible
    birational to $X$. Hence we find isomorphic open subset $U \subset X$, $V \subset \widetilde{X}$ which gives us
    \[
        [X \setminus U ] - [\widetilde{X} \setminus V] = [X] - [U] - [\widetilde{X}] + [V] = [X] - [\widetilde{X}].
    \]
    Thus $[X]$ can be written as the sum of $[\widetilde{X}]$ and some varieties of lower dimension for which we can invoke the induction
    hypothesis.
\end{proof}

We are going to use Theorem \ref{bittner} in the following form we proved in the induction step.

\begin{corollary}
    \label{decomp}
    For each (irreducible) variety $X$ over a field of characteristic zero there exists a smooth, irreducible, projective variety $Y$ birational
    to $X$ such that $[X] = [Y] + \sum_i m_i[W_i]$ where the $[W_i]$ are smooth, irreducible, projective varieties of dimension strictly
    smaller than the dimension of $X$. \qed
\end{corollary}



%XXX: More examples?
%TODO: symmetric product
%TODO: Example: symmetric product of P_k^1
%TODO: Motivic zeta function
%TODO: Examples for zeta functions (at least \zeta(P^1)
%XXX: keep the _k subscripts when writing A_k^n and P_k^n?

\section{Symmetric Products and the Motivic Zeta Function}
\label{symProd}
We now want to introduce the concept of the symmetric product, i.e. how it can be defined as the quotient of the (regular) product by the
action of the symmetric group. This section follows the presentation in \cite[Appendix A]{mustata} and \cite[Lecture 10]{harris}.
%%TODO: Appendix -> App.? Lecture ->?
\begin{definition}
    Let $X$ be an $S$-Scheme, $G$ a group acting on $X$ (from the left) 
    via $S$-automorphisms. Then a variety $Y$ together with a morphism $\pi \colon X \to Y$
    over $S$ is called the 
    \emph{categorial quotient} (also denoted by $X/G$) of $Y$ by $G$ if $\pi$ is $G$-invariant which means that for all $g \in G$ we have
    $\pi \circ g = \pi$ and is universal in this respect, i.e. for every other $G$-invariant morphism $f \colon X \to Z$ there exists a unique
    $G$-invariant morphism $g \colon Z \to Y$that makes the following diagram commute
    \begin{equation*}
        \begin{tikzcd}
            X \arrow{r}{\pi} \arrow{d}{f} & X/G \\
            Z \arrow[dashleftarrow]{ur}{g}
        \end{tikzcd}.
    \end{equation*}
    Then $\pi \colon X \to Y$ is the categorial quotient of $X$ by the action of $G$.
\end{definition}

\begin{remark}
    Since it is defined via an universal property, the quotient is unique up to unique isomorphism if it exists. 
    In contrast to other settings the scheme theoretic quotient does not always exist and can be quite subtle to construct.
    In the following we construct the quotient in the case that $G$ is finite and the scheme $X$ is a quasiprojective variety.
\end{remark}

We first construct the quotient in the affine case. If we were actually taking the quotient in the category of affine schemes things would be
quite simple: we could as well work in the category of rings and construct something that might be called a co-quotient, i.e.\ an object that
satisfies the conditions of the definition, just with all arrows reversed (now $G$ is a right-action). This is simply the $G$ invariant ring
$A^G \coloneqq \{a \in A | ga = a \ \forall g \in G\} \subset A$. Hence in the category of affine schemes the quotient would be given by 
$\Spec(A) \to \Spec(A^G)$. But since we would like to take quotients in the full category of schemes we have to do some more work.
First we slightly rephrase the definition for the case of schemes.
\begin{lemma}
    \label{quotProps}
    Let $\pi \colon X \to Y$ be a surjective morphism of schemes together with a group $G$ acting on $X$ via automorphism 
    such that the following conditions hold:
    \begin{enumerate}[label=\rm{\roman*)}]
        \item The fibres of $\pi$ are in one to one correspondence to the orbits of the $G$-action.
        \item $Y$ carries the quotient topology.
        \item There exists a unique isomorphism $\mathcal{O}_X = \pi_*\mathcal{O}_Y^G$
    \end{enumerate}
\end{lemma}
\begin{proof}
    The first two properties ensure that $\pi$ is a quotient of topological spaces, i.e. $\pi$ is a universally $G$-invariant continous map.
    By the discussion above, the third condition ensures that now actually all these morphisms are also morphism of locally ringed spaces.
\end{proof}

\begin{remark}
    \label{quotLocal}
    This shows that the quotient is in fact a local notion in the sense that if $U$ is an open subset of the quotient $X/G$ 
    then $\pi|_{\pi^{-1}(U)} \colon \pi^{-1}(U) \to U$ is the quotient of $\pi^{-1}(U)$ by the action of $G$.
\end{remark}

\begin{lemma}
    \label{quotAffine}
    Let $X = \Spec{A}$ be an affine variety with a finite group $G$ acting on it. Then the categorial quotient exists, is itself a variety and
    is given by $X/G = \Spec(A^G)$, the spectrum of the $G$-invariant subring of $A$.
\end{lemma}
\begin{proof}
    We first show that $\Spec(A^G)$ is again a variety. As a subring of a reduced $k$-algebra $A^G$ is again reduced, so what is left to check
    is that it is of finite type over $k$. Let $T_1,\dots,T_n$ be generators of $A$ as $k$-algebra. By passing to the union of the orbits of
    the $T_i$ we might assume that $G$ acts on the set of generators via permutations and hence can be seen as a subgroup of the 
    symmetric group $S_n$. Hence we get the following chain of ring inclusions: $k \subset A^{S_n} \subset A^G \subset A$. Now $A^{S_n}$ is
    generated by the elementary symmetric polynomials in the $T_i$ and hence is of finite type over $k$.
    It is now enough to show that $A$ is integral over $A^{S_n}$ because then $A$ is of finite type and integral over $A^{S_n}$ and hence
    finite, and since $A^{S_n}$ is noetheriean $A^G$, as a submodule of $A$, 
    is again a finite module over $A^{S_n}$ hence of finite type over $k$.
    So let be $a \in A$, then we find a $f_a \in k[X_1,\dots,X_n]$ such that $f_a(T_1,\dots,T_n) = a$, and we find that $a$ is a zero of
    the polynomial
    \[
        \prod_{g \in S_n} (X - g(a)) = \prod_{g \in G} (X - f_a(g(T_1),\dots,g(T_n))) \in A^{S_n}[X]
    \]
    hence it is integral over $A^{S_n}$ (and also over $A^G$).
    This proves that $\Spec(A^G)$ is again a variety. To show that it is also the quotient of $X$ by the action of $X$ we check the properties
    i) to iii) of \ref{quotProps}. Let us prove the first property.
    Since we have $g\mathfrak{p} \cap A^G$ = $\mathfrak{p} \cap A^G$ each orbit is contained in some fibre. For the converse assume we have
    prime ideals $\mathfrak{p},\mathfrak{q} \subset A$ such that $\mathfrak{p} \cap A^G = \mathfrak{q} \cap A^G$.
    For $a \in \mathfrak{p}$ we have $\prod_{g \in G} ga \in \mathfrak{p} \cap A^G = \mathfrak{q} \cap A^G$, and since $\mathfrak{q} \cap A^G$
    is prime we find some $g \in G$ such that $ga \in \mathfrak{q} \cap A^G$ or equivalently 
    $a \in g^{-1}\mathfrak{q} \cap A^G \subset \mathfrak{q}$, hence 
    we find $\mathfrak{p} \subset \bigcup_{g \in G} g\mathfrak{q}$ and can apply the prime avoidance lemma to find that $\mathfrak{p}$ had to be
    contained in some $g\mathfrak{q}$. Because the situation is symmetric in $\mathfrak{p}$ and $\mathfrak{q}$ we find $h \in G$ such that
    $\mathfrak{q} \subset h\mathfrak{p}$ hence $\mathfrak{p} \subset g\mathfrak{q} \subset hg\mathfrak{p}$. Now we use the fact that
    for a integral extension two prime ideals with the same contraction with one subset of another must be equal. Applied to this
    situation and again using that $\mathfrak{p} \cap A^G = hg\mathfrak{p} \cap A^G$ we obtain $hg\mathfrak{p} = \mathfrak{p}$ and by the same
    argument finally $\mathfrak{p} = g\mathfrak{q}$. Hence we have shown that fibres and orbits coincide. Furthermore this shows that $\pi$ is
    surjective and finite, hence $\Spec(A^G)$ has the quotient topology.
    For the last property we note that the action of $G$ extends to an action on $A_f$ for some $f \in A^G$ and we have 
    $A^G_f = (A_f)^G$ for some $f \in A^G$ since $\frac{a}{f^n} = \frac{ga}{f^n}$ if and only if $a = ga$. This means that the natural
    map $A^G_f = \mathcal{O}_{Spec(A^G)}(D(f)) \to \pi_*\mathcal{O}_{Spec(A)}(D(f))^G = (A_f)^G$ is an isomorphism, and since the $D(f)$ form
    a basis of the topology we hav proven property iii).
\end{proof}

\begin{remark}
    Using the same reasoning as above we can conclude that if $X = \Proj(B)$ for some finite type graded $k$-algebra $B$, 
    and the action of $G$ on $X$ comes from
    an action of $G$ on $B$ via automorphisms of graded $k$-algebras, then we may construct the quotient directly as $X/G = \Proj(B^G)$.
\end{remark}

We now turn to a relevant special case: the quotient of an quasiprojective variety $X$ by the action of a finite group $G$, using the following
general result.

\begin{lemma}[{\cite[Prop. 1.8]{SGA1}}]    
    Let $X$ be a scheme with a finite group $G$ acting on it via automorphisms. If there exists an open cover by $G$-invariant affine sets
    then the quotient $X/G$ exists.
    We call such an action \emph{admissible}
\end{lemma}
\begin{proof}
    Say $X$ is covered by the $G$-invariants affine sets $U_1,\dots,U_i$ for which we may construct the quotient $\pi_i \colon U_i \to U_i/G$ 
    as by the previous lemma. Now we would like to glue these. This is possible since by Remark \ref{quotLocal} 
$\pi_i \colon U_i \cap U_j \to \pi_i(U_i \cap U_j)$ and $\pi_j \colon U_i \cap U_j \to \pi_j(U_i \cap U_j)$ are both quotients of $U_i \cap U_j$
    by $G$ and therefore naturally isomorphic. Since we may check the conditions of Lemma \ref{quotProps} locally gluing gives a quotient of 
    $X$ by $G$.
\end{proof}

\begin{corollary}
    If $X$ is a quasiprojective variety, then the quotient by the action of a finite group always exists and is itself a quasiprojective variety.
\end{corollary}
\begin{proof}
    Let $X \subset \mathbb{P}_k^n$ be locally closed with a finite group $G = \{g_1,\cdots,g_m\}$ acting on it. 
    Consider the orbit $\{g_1(x),\cdots,g_m(x)\} \subset X$ of a point $x \in X$, and denote by $\mathfrak{p}_1,\cdots,\mathfrak{p}_m$ the 
    corresponding graded prime ideals corresponding to $\overline{\{x\}}$, and by $\mathfrak{q}$ the graded prime ideal corresponding to
    $\overline{X} \setminus X$ (taken to be the irrelevant ideal if this set is empty). 
    Hence the ideal $\mathfrak{q}$ is not contained in any of the $\mathfrak{p}_i$s hence, by the graded version of the prime
    avoidance lemma we find an element with positive degree $y \in \mathfrak{q}$ that is contained in none of the $\mathfrak{p}_i$ and hence
    the hypersurface $V_+(y)$ contains $\bar{X} \setminus X$ but not the $g_1(x),\cdots,g_m(x)$, so $U \coloneqq D_+(y)$ 
    is an affine set that lies in $X$ and contains the orbit of $x$ under $G$. Now consider $U' \coloneqq \bigcap_{g \in G} g(U)$. This is clearly
    $G$-invariant, and since $U$ contains the orbit of $y$ we still find that $y$ lies in $U'$. Hence we have found a open cover of X by
    $G$-invariant open affines, and thus the quotient exists by the above lemma.
\end{proof}

\begin{definition}
    \label{symdef}
    Let $X$ be a quasiprojective variety, then the $n$-fold \emph{symmetric product} $\Sym^n(X)$ is defined as the quotient of $X^n$
    (which is again quasiprojective via the Segre embedding) by the natural action of the symmetric group in $n$ letters.
\end{definition}

\begin{example}
    By Lemma \ref{quotAffine} we have $Sym^n(\mathbb{A}^1) = \mathbb{A}^n/S_n = \Spec(k[x_1,\dots,x_n]^{S_n}).$ Where $S_n$ acts on the
    polynomial ring by permuting the $x_i$. As remarked above, by the fundamental theorem of elementary symmetric functions the subring
    of symmetric polynomials is generated as an algebra by the symmetric polynomials and these are algebraically independeent, hence
    $Sym^n(\mathbb{A}^1) \cong \mathbb{A}^n)$.
\end{example}

\begin{example}
    The symmetric product of a variety $X$ need not be smooth even though $X$ is smooth. 
    Indeed, consider $Sym^2(\mathbb{A}_k^2) = \Spec(k[x_1,x_2]))$. Suppose $char k \neq 2$, 
    for convenience we also assume $k$ algebraically closed.
    The action on $\mathbb{A}^4$ is given by exchanging points, so we may see this as linear action by the subgroup of $GL_n$ generated
    by
    \[
    \begin{pmatrix}
        0 & 0 & 1 & 0\\
        0 & 0 & 0 & 1\\
        1 & 0 & 0 & 0\\
        0 & 1 & 0 & 0\\
    \end{pmatrix}
    \]
    By changing the basis to
    \[
        \begin{pmatrix} 1\\ 0\\ 1\\ 0 \end{pmatrix},
        \begin{pmatrix} 0\\ 1\\ 0\\ 0 \end{pmatrix},
        \begin{pmatrix} 1\\ 0\\ -1\\ 0 \end{pmatrix},
        \begin{pmatrix} 0\\ 1\\ 0 \\ -1\end{pmatrix}
    \]
    $S_2$ acts on the first two basis vectors as the identity and the last two are sent to the negative of themselves. Hence we can decompose
    up $k^4$ in two $S_2$ invariant subspaces and all is left is to figure out what the quotient of $\mathbb{A}^2$ by the action
    $\mathbb{A}^2 \to \mathbb{A}^2,\ (a,b) \mapsto (-a,-b)$ is. The action on the coordinate ring is given by $x \mapsto -x, \ y \mapsto \-y$.
    The polynomials invariant under this action are the polynomials which only have monomials with even degree.
    So the invariant ring is generated by $x^2$, $xy$ and $y^2$ and we claim that the kernel of the map
    \begin{gather*}
        \pi \colon k[u,v,w] \to k[x,y]^{S_2}\\
        u \mapsto x^2,\ 
        v \mapsto y^2,\ 
        w \mapsto xy
    \end{gather*}
    is given by $(uv - w^2)$. Clearly $uv - w^2$ is in the kernel, so suppose $f \in \mathrm{ker}(\pi)$. 
    Polynomial division in $k[u,v][w]$ yields
    $f = g(uv - w^2) + r$, since $r$ is of strictly smaller degree than $uv - w^2$ it is of the form $r = wh + h'$, $h, h' \in k[u,v]$.
    Thus $0 = \pi(r) = \pi(wh + h') = xy\pi(h) + \pi(h')$. 
    Note that $\pi(h)$ and $\pi(h')$ are polynomials in $x^2$ and $y^2$ so the degree in $x$ and $y$ is odd for $xy\pi(h)$ and even for $\pi(h')$,
    hence equality can only hold if $h = h' = 0$. Thus $r$ has to be zero and $f$ lies in $(uv - w^2)$.
    Putting it all togheter, $Sym^2(\mathbb{A}^2) \cong k[x_1,x_2,x_3,x_4,x_5]/(x_3x_4 - x_5^2)$. This is singular in the origin by the Jacobi
    criterion.
\end{example}

\begin{example}
    $Sym^n(\mathbb{P}^1) = \mathbb{P}^n$. 
    Let us first recall some facts about the product of projective varieties. If $X = Proj(B)$ for some graded $k$-algebra of finite type,
    then $X^n = Proj(\bigoplus_{d \ge 0} \bigotimes_{i = 0}^n B_d$. The action of 
\end{example}

\begin{definition}
    Let $\mu \colon \gring \to A$ be a motivic measure and $X$ a quasiprojective variety. Then we define the \emph{motivic zeta function} 
    $\zeta_{\mu}(X,t) \in A[[t]]$ as
    \[
        \zeta_{\mu}(X,t) \coloneqq \sum_{n=0}^\infty \mu([\Sym^n(X)])t^n
    \]
\end{definition}

\begin{example}
    We can now compute a first zeta function, namely $\zeta_{id}(\mathbb{P}_k^1, t)$. Using the identity and notation from example \ref{projSum}
    we calculate
    \begin{align*}
        \zeta_{id}(\mathbb{P}_k^1, t) & = \sum_{n=0}^\infty \left[\Sym^n({\mathbb{P}_k^1})\right]t^n 
                                       = \sum_{n=0}^\infty \left(\sum_{k=0}^n \mathbb{L}^k\right) t^n \\
                                       & = \sum_{n=0}^\infty \sum_{k=0}^n \left(\mathbb{L}t\right)^k t^{n-k} \\
                                      & = \left( \sum_{n=0}^\infty t^n \right) \left( \sum_{n=0}^\infty (\mathbb{L}t)^n \right) 
                                        = \frac{1}{(1-t)(1-\mathbb{L}t)} .%TODO: add a fullstop that does not look completely shitty
    \end{align*}
    Hence $\zeta_{id}(\mathbb{P}_k^1, t)$ is in fact a rational function.
\end{example}

\textbf{The following is HIGHLY EXPERIMENTAL}

After discussing the genus zero case, we now sketch the picture for arbitrary smooth projective curves. 
For simplicity we will assume that our ground field $k$ is algebraically closed, but
this works in general whenever the curve in question has a rational point. 
So let $C$ be a smooth projective connected curve of genus $g > 0$ with a distinguished (rational) point $P \in C$.

\begin{theorem}%TODO: explain what Pic^0(X) is as a variety!!!
    Let $n$ be greater than $2g-2$, then the map 
    \begin{align*}
        Sym^n(C) &\to Pic^0(X) \\
        [x_1,\dots,x_n] &\mapsto \mathcal{L}(x_1 + \dots + x_n - nP)
    \end{align*}
    Is a projective bundle of rank $n - g$ %XXX: what is the convention here? n-g or n-g+1?
\end{theorem}
\begin{proof}
    We only sketch the idea, for a complete proof and a rigourous introduction of the employed concepts see (TODO: Milne, Mumford?)
    The fibre over some $\mathcal{L}(D)$, for $D = \sum x_i - nP$ is the complete linear system $|D| = \mathbb{P}(H^0(C, \mathcal{L}(D)))$.
    As usual, denote by $l(D)$ the dimension of the global sections of $\mathcal{L}(D)$, then Riemann-Roch tells us 
    that $l(D) - l(K-D) = deg(D) - g + 1 = n - g + 1$, but since $deg(K) = 2g-2$, for $n > 2g-2$ we find $l(K-D) = 0$, and hence
    $l(d) = n-g+1$, so the fibre is indeed isomorphic to $\mathbb{P}^(n-g)$. It of course remains to show that the morphism does in
    fact locally look like the projection of $Pic^0(X) \times \mathbb{P}^{n-g)}$ onto the first factor.
\end{proof}

Using this in conjunction with Lemma TODO we can obtain a nice closed form for the zeta function of an eliptic curve.
Recall that an elliptic curve is a smooth projective connected curve $E$ of genus $1$ with a distinguished rational point $P \in E$, the
the map $E \to Pic^0(E), x \mapsto L(x-P)$ is then an isomorphism. By the above theorem the map $Sym^n(E) \to Pic^0(X)$ is a projective bundle for
$n > 2g - 2 = 0$, i.e.\ it locally looks like the projection of $Pic^0(E) \times \mathbb{P}^n-1$ onto the first factor.
Thus by Lemma TODO in the Grothendieck ring we have for $n>0$ 
\begin{equation}
    [Sym^n(E)] = [Pic^0(E)][\mathbb{P}^{n-1}] = [E][\mathbb{P}^{n-1}]
\end{equation}
Hence we calculate
\begin{align*}
    \zeta_{\mathrm{id}}(E,t) &= \sum_{n=0}^{\infty} [Sym^n(E)] = 1 + \sum_{n=1}^\infty [E][\mathbb{P}^{n-1}]t^n \\
                             &= 1 + [E]t \sum_{n=1}[\mathbb{P}^{n-1}]t^{n-1} = 1 + [E]t \zeta_{\mathrm{id}}(\mathbb{P}^1, t)\\
                             &= 1 + \frac{[E]t}{(1-t)(1-\mathbb{L}t)} = \frac{1 + ([E]-[\mathbb{P}^1])t + \mathbb{L}t^2}{(1-t)(1-\mathbb{L}t)}.
\end{align*}

In the general case of positive genus we still get
\begin{align*}
    \zeta_{\mathrm{id}}(C,t) &= \sum_{n=0}^{2g-2} [Sym^n(C)] + \sum_{n=2g-1}^\infty [Pic^0(C)][\mathbb{P}^{n-1}]t^n\\
                             &= \sum_{n=0}^{2g-2} [Sym^n(C)] + [Pic^0(C)]t\zeta_{\mathrm{id}}(\mathbb{P}^1, t).
\end{align*}
And thus we have shown that $\zeta_{\mathrm{id}}(C,t)$ is a rational function.

\section{Constructing Motivic Measures}
\label{const}
We will now construct the measure $\mu_h$ of Theorem \ref{irrational}.

The first important result on the way is a lemma that helps us to construct such measures by extending maps from the monoid of smooth, irreducible
and complete varieties. For this we quickly recall the notion of a monoid ring.

\begin{definition}
    Let $G$ be a (multiplicative) monoid. Write $\mathbb{Z}[G]$ for the free abelian group over $G$, i.e. the set of formal sums
    $\sum_{g \in G} n_g g$ with the $m_g$ being integers, only finitely many non zero.
    Defining multiplication as
    \[
        \Big(\sum_{g \in G} n_g g\Big)\Big(\sum_{g \in G} m_g g \Big) = \sum_{g \in G} \Big(\sum_{hl = g} n_h m_l\Big) g
    \]
    this forms a ring, the \emph{monoid ring} over $G$.
\end{definition}

\begin{theorem}
    \label{th1}
    Set $k = \mathbb{C}$. Let $G$ be an abelian monoid and $\mathbb{Z}[G]$ the corresponding monoid ring. As above, denote
    by $\mathcal{M}$ the multiplicative monoid of irreducible smooth complete varieties. Let
    \[
        \psi \colon \mathcal{M} \to G
    \]
    be a homomorphism of monoids such that
    \begin{enumerate}[label=\rm{\roman*)}]
        \item $\psi([X]) = \psi([Y])$ if $X$ and $Y$ are birational;
        \item $\psi([\mathbb{P}^n]) = 1$ for all $n \ge 0$.
    \end{enumerate}
    Then $\psi$ can be uniquely extended to a ring homomorphism 
    \[
        \phi \colon \gring[\mathbb{C}] \to \mathbb{Z} [G]
    \]
\end{theorem}

%XXX: maybe rewrite the proof in a more "hands on" fashion to make clear why we check what we check and how \phi is defined
%XXX: why is \phi multiplicative?
\begin{proof}
    By Theorem \ref{bittner} we know that the Grothendieck ring (as an abelian group) is generated by $\mathcal{M}$, so after extending $\psi$ to
    the free abelian group over $\mathcal{M}$ 
    we have to check that $\psi$ preserves the relations of the blowup presentation of $\gring[\mathbb{C}]$ and hence factors over it,
    i.e.\ given a blow-up $f \colon X \to Y$ with smooth center $Z \subset X$ then $\psi([X]) - \psi([f^{-1}(Z)]) = \psi([Y]) - \psi([Z])$. 
    Note that $[X]$ and $[Y]$ are birational, since $f$ is a blowup, thus $\psi([X]) = \psi([Y])$ by property i).
    
    Now, since $X$ and $Z$ are nonsingular, the exceptional divisor $f^{-1}(Z)$ is isomorphic to 
    the projective space bundle  $\mathbb{P}(\mathcal{I}/\mathcal{I}^2)$ 
    where $\mathcal{I}$ is the ideal sheaf associated to $Z$ (see for example \cite[II.8, Thm 8.24]{Ha}),
    which is a $\mathbb{P}^n$-bundle, i.e.\ locally trivializes to $\mathbb{P}_k^n \times Z$
    and hence is birational to $\mathbb{P}_k^n \times Z$. 
    We can thus write $\psi([f^{-1}(Z)])$ as  
    \[
        \psi([f^{-1}(Z)]) = \psi([Z \times \mathbb{P}^n]) = \psi([Z][\mathbb{P}^n]) = \psi([Z])\psi([\mathbb{P}^n]) = \psi([Z])
    \]
    where the last equation holds because of property ii).
    Hence, we can extend $\psi$ to define a morphism $\phi \colon \gring[\mathbb{C}] \to \mathbb{Z} [G]$
\end{proof}

Before constructing a suitable monoid homomorphism that will yield our desired measure, we recall the definition of the Hodge numbers
of a variety as well as two basic results (all these are excercises in \cite[Ch. II]{Ha}).

\begin{definition}
    Let $X$ be a projective variety. The \emph{Hodge number} $h^{i,j}$ is given as the dimension of $H^j(X,\Omega^i_X)$, 
    which is a finite dimensional $k$-vector space by a theorem of Serre (\cite[§3 Prop. 7]{FAC}).
    %XXX: why not directly define for 
    If $d$ is the dimension of a nonsingular variety $X$ we will also write $P_g(X)$ for $h^{d,0}(X)$, called the \emph{geometric genus} of $X$.
\end{definition}


\begin{lemma}[{\cite[II Ex. 8.8]{Ha}}]
    \label{bir}
    The Hodge numbers $h^{i,0}$ are birational invariants of a variety: let $X, Y$ be birationally equivalent, smooth, projective varieties
    then $h^{i,0}(X) = h^{i,0}(Y)$.
\end{lemma}

Let us compute the hodge numbers of the projective space over a field as an example.
Recall that we have the euler exact sequence 
\[0 \to \Omega_{\mathbb{P}_k^n} \to \mathcal{O}_{\mathbb{P}_k^n}^{\oplus n+1}(-1) \to \mathcal{O}_{\mathbb{P}_k^n} \to 0 \]

We also need the following general lemma.

\begin{lemma}
    Let 
    \[
        0 \to \mathcal{F}' \to \mathcal{F} \to \mathcal{F}'' \to 0
    \]
    be a short exact sequence of locally free sheaves and assume furthermore that $\mathcal{F}''$ is of rank one.
    Then for any $p \ge 1$ the sequence
    \[
        0 \to \Lambda^p \mathcal{F}' \to \Lambda^p \mathcal{F} \to \Lambda^{p-1} \mathcal{F}' \otimes \mathcal{F}'' \to 0
    \]
    is exact.
\end{lemma}

\begin{lemma}[Chow, {\cite[II Ex. 4.10]{Ha}}]
    Every proper variety is birational to a projective variety.
\end{lemma}

\begin{remark}
    By \cite[Prop. 10.1 (d)]{Ha} the product of two smooth varieties over $k$ is again smooth. Hence the isomorphism classes of smooth
    irreducible complete varieties form a multiplicative monoid, in the following denoted by $\mathcal{M}$.
\end{remark}
\begin{definition}
    Denote by $C \subset \mathbb{Z}[t]$ the multiplicative monoid of polynomials with positive leading coefficient.
    For a smooth projective complex variety $Z$ of dimension $d$ define
    \[
        \Psi_h(X) \coloneqq1 + h^{1,0}(X)t + \cdots + h^{d,0}(X)t^d \in C.
    \]
    (Note that the Hodge numbers might be zero, but $\Psi_h(X)$ still has a positive leading coefficient, 
    just its degree might be smaller than the dimension of $X$)
    By Chow's lemma we can also define $\Psi_h$ for a smooth complete variety $Z$ by choosing a smooth projective variety $X$ which is birational
    to $Z$ and setting $\Psi_h(Z) \coloneqq \Psi_h(X)$. This is well defined by Lemma \ref{bir}.
\end{definition}

We now check that $\Psi_h$ satisfies all conditions of Theorem \ref{th1}. Independence of birational equivalence class was the content of Lemma
\ref{bir}. 

To check multiplicativity, we use the following lemma.

\begin{lemma}
    Let $X,Y$ be smooth, irreducible, projective $k$-varieties. Then the following equality holds:
    \[
        h^{p,0}(X \times_k Y) = \sum_{i+j=p} h^{i,0}(X)h^{j,0}(Y)
    \]
\end{lemma}
\begin{proof} TODO \end{proof}

With this we directly calculate 
\begin{align*}
    \Psi_h(X)\Psi_h(Y) &= \left( \sum_n h^{n,0}(X)t^n \right) \left( \sum_n h^{n,0}(Y)t^n \right)  
    = \sum_n \left( \sum_{i+j=n} h^{i,0}(X) h^{j,0}(Y) \right )t^n \\ &= \sum_n h^{n,0}(X \times_k Y)t^n = \Psi_h(X \times_k Y)
\end{align*}

%TODO: Check \Psi_h(P_k^n) = 1 (use Euler sequence)

\section{Irrationality of the Motivic Zeta Function}
\label{final}
The constructed motivic measure does not yet take values in a field, so we would like to pass to the quotient field of $\mathbb{Z}[\mathcal{M}]$.
That we are able to do so is the content of the next lemma.

\begin{lemma}
    Let $A$ be a factorial ring, and $S \subset A$ a multiplicative submonoid such that $1$ is the only unit in $S$. Then the monoid ring
    $\mathbb{Z}[S]$ is a polynomial ring (in possibly infinitely many variables), and hence an integral domain.
\end{lemma}
\begin{proof}
    Since $A$ is factorial, every $s \in S$ has a unique factorization, and since 1 is the only unit in $S$, 
    $s$ can be uniquely written as product of prime elements, hence if we take $B$ to be the polynomial ring over the formal
    variables $\{x_s | s \in S, s\ \text{prime}\}$ we get an isomorphism of rings
    \[
        B \to \mathbb{Z}[S], \ x_s \mapsto s
    \]
\end{proof}

\begin{definition}
    Denote by $\mathcal{H}$ the quotient field of $\mathbb{Z}[C]$ where, as above, $C$ denotes the submonoid of $\mathbb{Z}[t]$ 
    consisting of polynomials with positive leading coefficients. Since $-1$ is not contained in $C$ it satisfies the conditions of the
    previous lemma.
    We define the motivic measure $\mu_h \colon \gring \to \mathcal{H}$ as the measure obtained extending $\Psi_h$ as by Theorem \ref{th1}
\end{definition}

\begin{lemma}
    \label{same}
    Let $Y_1,\dots,Y_s,Z$ be irreducible varieties of dimension d over a field of characteristic zero 
    such that $\mu_h([Z]) = \sum_i n_i \mu_h([Y_i])$ for some $n_i \in \mathbb{Z}$ and $P_g(Z) \neq 0$ then $P_g(Z) = P_g(Y_i)$ for some $i$
\end{lemma}
\begin{proof}
    By Corollary \ref{decomp}, we find smooth, irreducible projective varieties $\overline{Z},\overline{Y_1},\cdots, \overline{Y_s}$ in the same
    birational class as the original varieties allowing us to rewrite the original equality as 
    \[
        \mu_h([\overline{Z}]) = \sum n_i\mu_h([\overline{Y}_i]) + \sum l_i \mu_h([X_i])
    \]
    where the $X_i$ are smooth irreducible varieties of dimension $<d$.
    Now, since $\mu_h$ was obtained as an extension of $\Psi_h$ which was defined on smooth, irreducible, projective varieties, this is
    actually an equation in $\mathbb{Z}[C]$, namely
    \[
        \Psi_h(\overline{Z}) = \sum n_i \Psi_h(\overline{Y}_i) + \sum l_i\Psi_h(X_i)
    \]
    Since there are no (additive) relations between elements of $C$ in the monoid ring, one of the polynomials $\Psi_h(\dots)$ on the 
    right hand side, must actually be the polynomial on the left hand side, but with the dimension of 
    $X_i$ being strictly smaller than that of $Z$ and the assumption that
    $P_g(\overline{Z}) = h^{d,0}(\overline{Z}) \neq 0$ it cannot be one of the $\Psi_h(X_i)$ because all these polynomials have strictly smaller
    degree. Hence we have $\Psi_h(\overline{Z}) = \Psi_h(\overline{Y}_i)$ for some $i$ and in particular, since they are of the same dimension, 
    the genus of $\overline{Z}$ and $Y_i$ must agree.
\end{proof}

Now let $X$ be a smooth projective surface with $P_g(X) \geq 2$. We will show that $\zeta_{\mu_h}(X,t) \in \mathcal{H}[[t]]$ is not rational,
thus proving Theorem \ref{irrational}. We need one technical lemma about the genus of symmetric powers of X whose proof is out of scope of this
thesis.

\begin{lemma}[{\cite[Lem. 3.8]{MR1996804}}]
    \label{genus}
    Let $X$ be a smooth, projective surface over $\mathbb{C}$. Then
    \[
        P_g(\Sym^n(X)) = \binom{P_g(X) + n - 1}{P_g(X) - 1}
    \]
\end{lemma}

We will use the following rationality criterion for power series.

\begin{lemma}[{\cite[Lem.3.1]{bruhat}}]
    Let $k$ be a field and suppose the power series $\sum a_it^i \in k[[t]]$  is a rational function (i.e.\ an element in $k(t)$) then
    there exist $n, n_0 > 0$ such that for each $m > n_0$ the determinant
    \[
        \begin{vmatrix}
            a_m     & a_{m+1} & \dots & a_{m+n} \\
            a_{m+1} & a_{m+2} & \dots & a_{m+n+1} \\
            \vdots  & \vdots  & \ddots & \vdots \\
            a_{m+n} & a_{m+n+1} & \dots & a_{m+2n}
        \end{vmatrix}
    \]
    vanishes.
\end{lemma}
\begin{proof}
    Denote the above matrix by $A$. Since $\sum a_it^i$ is rational, we may write it as
    \[
        \sum a_it^i = \frac{p_0 + \dots + p_et^e}{q_0 + \dots + q_d t^n}
    \]
    with not all $q_i$ being zero.
    Hence 
    $p_0 + \dots + p_et^e = \left(\sum a_it^i\right)\left(q_0 + \dots + q_d t^d\right) = \sum_{i=0}^{\infty} \left(\sum_{k+l=i} a_kq_l\right)t^i$.
    This means that for $m > e$ we get $0 = \sum_{k+l=m} a_kq_l$. now note that $q_l = 0$ for $l > n$ hence we may rewrite this as
    \[
        0 = \sum_{l = 0}^{n} q_la_{m-l} = \sum_{l=0}^n q_{n-l}a_{m-n+l}.
    \]
    Thus we have shown that
    \[
        \begin{pmatrix}
            a_m     & a_{m+1} & \dots & a_{m+n} \\
            a_{m+1} & a_{m+2} & \dots & a_{m+n+1} \\
            \vdots  & \vdots  & \ddots & \vdots \\
            a_{m+n} & a_{m+n+1} & \dots & a_{m+2n}
        \end{pmatrix}
        \begin{pmatrix}
            q_n \\ \vdots \\ q_1 \\ q_0
        \end{pmatrix}
        =
        \begin{pmatrix}
            \sum_{l=0} a_{m+l}q_{n-l} \\ \vdots \\ \vdots \\ \sum_{l=0} a_{m+n+l}q_{n-l}
        \end{pmatrix}
        = 0
    \]
    for m big enough. Since one of the $q_l$ is non zero this means that $A$ has nonempty kernel and hence its determinant vanishes.
\end{proof}

\begin{proof}[Proof of Theorem \ref{irrational}]
    Assume that $\zeta_{\mu_h}(X,t)$ is rational, hence by the above criterion there is an $n$ such that for $m$ big enough
    \begin{gather*}
        \sum_{\sigma \in S_{n+1}} \sgn(\sigma) \mu_h \Big(\left[ \prod_{i=1}^{n+1} \Sym^{m + i + \sigma(i) - 2}(X)\right]\Big)  = 0 \\ 
            \Leftrightarrow
            \mu_h \Big ( \left[ \prod_{i=0}^n \Sym^{m + 2i}(X) \right] \Big )  = - \sum_{\substack{\sigma \in S_{n+1} \\ \sigma \neq id}} 
            \sgn(\sigma) \mu_h \Big ( \left[\prod_{i=0}^n \Sym^{m + i + \sigma(i+1) - 1}(X) \right]\Big )
    \end{gather*}
    Now we can apply Lemma \ref{same} to conclude that there is a permutation $\sigma$, not the identity permutation, such that
    \[
        P_g \Big(\prod_{i=0}^n \Sym^{m+2i}(X)\Big) = P_g \Big( \prod_{i=0}^n \Sym^{m+i+\sigma(i+1) - 1}(X) \Big)
    \]
    By Lemma \ref{genus} (and using the fact that the genus is multiplicative), we get
    \[
        \prod_{i=0}^n \binom{P_g(X) + m + 2i - 1}{P_g(X) - 1} - \prod_{i=0}^n \binom{P_g(X) + m + i + \sigma(i+1) - 2}{P_g(X) - 1} = 0.
    \]
    By assumption $P_g(X) \geq 2$ hence the left hand side, considered as an polynomial in $m$, is not the zero polynomial. So by taking
    $m$ large enough we obtain a contradiction.
\end{proof}

\bibliography{mybib}{}
\bibliographystyle{alpha}
\end{document}
