\documentclass[a4paper]{article}

\usepackage[english]{babel}
\usepackage[utf8x]{inputenc}
\usepackage{amsmath, amsthm, amssymb}

%bibliography
\usepackage{cite}

%font
\usepackage[euler-digits, euler-hat-accent]{eulervm}
\usepackage{newpxtext}

\theoremstyle{plain}
\newtheorem{theorem}{Theorem}
\newtheorem{corollary}{Corollary}
\newtheorem{lemma}{Lemma}
\newtheorem{example}{Example}

\theoremstyle{definition}
\newtheorem{definition}{Definiton}
\newtheorem{remark}{Remark}

%frontpage
\usepackage{BA_Titelseite}

\author{Maximilian Rath}
\geburtsdatum{2. Januar 1995}
\geburtsort{Reutlingen}

\date{2.2.2016}
\betreuer{Prof. Dr. Daniel Huybrechts}
\institut{Mathematisches Institut}


\title{Motivic Zeta Function}
\ausarbeitungstyp{Bachelorarbeit Mathematik}

%I don't want to write the Grothendieck-Ring over and over
\newcommand{\gring}[1][k]{K_0[\mathcal{V}_#1]}


\begin{document}

\maketitle

\begin{definition}
    In the following the term \emph{$k$-variety} always means a separated, integral scheme of finite type over a field $k$.
    Denote by $\mathcal{V}_k$ the category of $k$-varieties.
\end{definition}

\begin{definition}
    Let $k$ be a Field. Consider the group of formal linear combinations of isomorphism-classes in $\mathcal{V}_k$.
    Setting $[X] \times [Y] := [X \times Y]$ makes this into a ring.
    The \emph{Grothendieck ring of varieties} $\gring$ is then obtained by modding out relations of the form
    \[
        [X] - [Y] = [X \setminus Y]
    \]
    Where Y is closed in X.

    A \emph{motivic measure} is a ringhomomorphism $\mu: \gring \to A$ into a ring $A$. The identity function
    $id: \gring \to \gring$ is called the \emph{universal motivic measure}.
\end{definition}

Let us now make some remarks about this ring.
\begin{remark}
    The Grothendieck ring of varieties is commutative as $X \times Y \cong Y \times X$ for two schemes $X$ and $Y$.
\end{remark}
\begin{remark}
    By \cite[Proposition 10.1 (d)]{Ha} the product of two smooth varieties over $k$ is again smooth. Hence the isomorphism classes of smooth
    irreducible complete varieties form a multiplicative monoid, in the following denoted by $\mathcal{M}$.
\end{remark}

\begin{example}
    Using the decomposition $\mathbb{P}_k^n = \mathbb{P}_k^{n-1} \coprod \mathbb{A}_k^n$ where $\mathbb{P}_k^{n-1}$ is closed in $\mathbb{P}_k^n$  
    we get $[\mathbb{P}_k^n] = [\mathbb{P}_k^{n-1}] + [\mathbb{A}_k^n]$ in the Grothendieck ring.
    Inductively this yields the identity 
    \[
        [\mathbb{P}_k^n] = \sum_{k=0}^n [\mathbb{A}_k^1]^k
    \]
    %XXX: more natural place for introducing this convention?
    We also denote the isomorphism class of the affine line as $\mathbb{L}$.
\end{example}
%XXX: More examples?
%TODO: symmetric product
%TODO: Motivic zeta function
%TODO: Examples for zeta functions (at least \zeta(P^1)

In their paper \cite{MR1996804} Larsen and Lunts prove the following result:

\begin{theorem}
    Assume that $k = \mathbb{C}$. There exists a field $\mathcal{H}$ and a motivic measure $\mu: \gring \to \mathcal{H}$ with the following
    property: if X is a smooth complex projective surface such that $P_g(X)=h^{2,0}(X) \ge 2$, then the zeta-function $\zeta_{\mu}(X,t)$
    is not rational.
\end{theorem}

The first important theorem on the way to prove this result is this:
%XXX: Maybe define monoid ring?
\begin{theorem}
    \label{th1}
    Set $k = \mathbb{C}$. Let $G$ be an abelian commutative monoid and $\mathbb{Z}[G]$ be the corresponding monoid ring. As above, denote
    by $\mathcal{M}$ the multiplicative monoid of irreducible smooth complete varieties. Let
    \[
        \psi: \mathcal{M} \to G
    \]
    be a homomorphism of monoids such that

    (i) $\psi([X]) = \psi([Y])$ if $X$ and $Y$ are birational;

    (ii) $\psi([\mathbb{P}^n]) = 1$ for all $n \ge 0$.

    Then $\psi$ can be uniquely extended to a ring homomorphism 
    \[
        \phi: \gring[\mathbb{C}] \to \mathbb{Z} [G]
    \]
\end{theorem}

To prove this result we will use a result by Bittner (see \cite[Theorem 3.1]{Bittner}).

%XXX: more about blowups?
%XXX: which formulation is nicer: bittner or LL (used here)
%XXX: switch order? first this theorem on structure of the grothendieck ring, then present the first main theorem
\begin{theorem}
    The Grothendieck group $\gring[\mathbb{C}]$ is generated by classes of smooth complete varieties subject to relations of the form
    \[
        [X] - [f^{-1}(Z)] = [Y] - [Z]
    \]
    where $X$,$Y$ are smooth complete varieties and $f: X \to Y$ is a morphism which is a blowup with a smooth center $Z \subset Y$.
\end{theorem}
%%XXX: resolve all those whys
\begin{proof}[Proof of theorem \ref{th1}]
    We have to check that $\psi$ preserves the above relations, i.e. that $\psi([X]) - \psi[f^{-1}(Z)] = \psi([Y]) - \psi([Z])$. 
    But $[X]$ and $[Y]$ are birational since $f$ is a blowup. (WHY?)
    Now $f^{-1}(Z)$ is birational to $Z \times \mathbb{P}^n$ (WHY??) and thus 
    \[
        \psi([f^{-1}(Z)]) = \psi([Z \times \mathbb{P}^n]) = \psi([Z][\mathbb{P}^n]) = \psi([Z])\psi([\mathbb{P}^n) = \psi([Z])
    \]
    Hence we can linearly extend $\psi$ to define the morphism $\phi: \gring[\mathbb{C}] \to \mathbb{Z} [G]$
\end{proof}

\bibliography{mybib}{}
\bibliographystyle{alpha}
\end{document}
