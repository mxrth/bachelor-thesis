\documentclass[a4paper]{article}

\usepackage[english]{babel}
\usepackage[utf8x]{inputenc}
\usepackage{amsmath, amsthm, amssymb}

%font
\usepackage[euler-digits, euler-hat-accent]{eulervm}
\usepackage{newpxtext}

\theoremstyle{plain}
\newtheorem{theorem}{Theorem}
\newtheorem{corollary}{Corollary}
\newtheorem{lemma}{Lemma}

\theoremstyle{definition}
\newtheorem{definition}{Definiton}

\title{Motivic Zeta Function}
\author{Maximilian Rath}



\begin{document}

\begin{definition}
    In the following the term \emph{$k$-variety} always means a separated, integral scheme of finite type over a field $k$.
    Denote by $\mathcal{V}_k$ the category of $k$-varieties.
\end{definition}

\begin{definition}
    Let $k$ be a Field. Consider the group of formal linear combinations of isomorphism-classes in $\mathcal{V}_k$.
    Setting $[X] \times [Y] := [X \times Y]$ makes this into a ring.
    The \emph{Grothendieck ring of varieties} $K_0[\mathcal{V}_k]$ is then obtained by modding out relations of the form
    \[
        [X] - [Y] = [X \setminus Y]
    \]
    Where Y is closed in X.
\end{definition}

\end{document}
