\documentclass[11pt, a4paper, german]{article}

\usepackage[english]{babel}
\usepackage[utf8x]{inputenc}
\usepackage{amsmath, amsthm, amssymb}
\usepackage{mathtools} %for \coloneqq
\usepackage{faktor}
\usepackage{tikz-cd} 
%bibliography
\usepackage[noadjust]{cite}
\usepackage{hyperref}
%font
%\usepackage[euler-digits, euler-hat-accent]{eulervm}
%\usepackage{newpxtext}

\theoremstyle{plain}
\newtheorem{theorem}{Theorem}[section]
\newtheorem{corollary}[theorem]{Corollary}
\newtheorem{lemma}[theorem]{Lemma}

\theoremstyle{definition}
\newtheorem{definition}[theorem]{Definiton}
\newtheorem{remark}[theorem]{Remark}
\newtheorem{example}[theorem]{Example}

%frontpage
\usepackage{BA_Titelseite}

\author{Maximilian Rath}
\geburtsdatum{2. Januar 1995}
\geburtsort{Reutlingen}

\date{2.2.2016}
\betreuer{Prof. Dr. Daniel Huybrechts}
\institut{Mathematisches Institut}


\title{Motivic Zeta Function}
\ausarbeitungstyp{Bachelorarbeit Mathematik}

%I don't want to write the Grothendieck-Ring over and over
\newcommand{\gring}[1][k]{K_0[\mathcal{V}_#1]}

\DeclareMathOperator{\spec}{Spec}
\DeclareMathOperator{\Sym}{Sym}
\DeclareMathOperator{\Proj}{Proj}
\begin{document}

\maketitle
\section{Introduction}

In 1949 Andr\'e Weil, in the paper \cite{weil1949}, formulated his famous conjectures
which predicted certain deep properties of the Hasse--Weil zeta function for a variety $X$ over a finite field $k = \mathbb{F}_q$. 
The zeta function $Z(X,t) \in \mathbb{Q}[[t]]$ is defined as
%XXX: spacing before last paren
\[
    Z(X,t) = \exp \big(\sum_{m=0}^\infty \frac{|X(\mathbb{F}_{q^m})|}{m} t^m \ \big).
\]
%TODO: explain what rational points over bigger fields mean, in terms of solving equations 

Dwork first established the rationality of the zeta function, the most basic of the three conjectures. 
Later, using his new scheme theoretic approach to algebraic geometry, Grothendieck gave another proof for the special case of 
smooth projective varieties.

%\begin{theorem}
%    Let $X$ be a non-singular projective variety over a finite field $k = \mathbb{F}_q$. Then Z(X,t) is a rational function of t.
%\end{theorem}

Grothendieck's new perspective was fundamental for the work of Deligne, who used these new methods to finally establish the 
last part of the conjectures, an analogue of the Riemann hypothesis, in 1974 (\cite{MR0340258}).
In \cite{kapranov} Kapranov generalized the Hasse-Weil zeta function, defining it not only for varieties over finite fields
but varieties in general. It turns out that in this more general setting the zeta function is not as well behaved in terms of rationality as
the "classical" Hasse--Weil zeta function and we will investigate this phenomena more closely in the case of complex surfaces, following the 
work of Larsen and Lunts \cite{MR1996804}.

We now want to motivate what a possible generalization of the Hasse-Weil zeta function for varieties over a arbitrary field $k$ could be.
This part therefore, in some places, sacrifices mathematical precision for the sake of brevity.  
We still give citations for most of the results and will of course develop everything that is important for the presentation and proof 
of our main result in later sections in full detail.

Two things do not generalize easily: the passing to bigger and bigger extensions of the ground field $k$ (what if $k$ is algebraically closed,
for example?), and counting $k$-rational points (again, if $k = \bar{k}$ there are, in general, a lot of $k$-rational points). 

%XXX this section might be too much to develop in detail
To see what could be generalizations for these phenomena, 
we give another equivalent definition of the Hasse--Weil zeta function in terms of effective 0-cycles. 
Recall that a 0-cycle on a variety $X$ is an element of the free abelian group over the closed 
points of $X$. Such a cycle $\alpha = \sum_i n_i x_i$ is called \emph{effective} if the $n_i$ are greater than or equal to zero.
The \emph{degree} of $\alpha$ is given by $\deg(\alpha) = \sum_i n_i \deg(x_i)$ 
(with $\deg(x_i)$ being the degree of the field extension $\kappa(x_i)/k$).

Using this notation we can rewrite the zeta function as follows 
\begin{equation}
    Z(X,t) = \sum_{\alpha} t^{\deg(\alpha)}.
\end{equation}
Where the sum ranges over all effective 0-cycles on $X$.
(For a proof of this (elementary) identity see for example Musta\c{t}\u{a}'s great notes on zeta functions in algebraic geometry 
\cite[Rem. 2.9]{mustata})

This definition still does not easily generalize, since for example $\mathbb{A}_{\mathbb{C}}^1$ has infinitely many 0-cycles of degree $n$.

But counting degree zero cycles of a given degree $n$, is the same as counting $k$-rational points of the $n$-fold symmetric product of the 
variety $X$ (denoted by $\Sym^n(X)$).
We will not formally introduce the symmetric product until later, but thinking of it as the quotient of 
the $n$-fold product of $X$ with itself by the natural action of the symmetric group in $n$ letters gives the right intuition. 
For $k$ algebraically closed, the bijection is obvious, since then the $k$-rational points are exactly the closed points. 
Dividing out by the action of the symmetric group hence introduces the commutativity-relations we have if we consider formal sums
of such closed points. This intuition carries over for the case of finite fields, hence we get

\begin{equation}
    Z(X,t) = \sum_{n} |\Sym^n(X)(k)| t^n.
\end{equation}

So, in some rough sense, we are "measuring" the size of bigger and bigger symmetric products of X, so we expect the symmetric product
to make an apearance in our generalized zeta function, taking the role of counting solutions over bigger and bigger field-extensions.


\section{The Grothendieck Ring of Varieties and Motivic Measures}

\begin{definition}
    In the following the term \emph{$k$-variety} always means a separated, reduced scheme of finite type over a field $k$.
    We will write $\mathcal{V}_k$ for the category of $k$-varieties.
\end{definition}

We now collect some properties of counting rational points of a variety over a finite field, that will guide our definition for "measures" 
of the symmetric product.

\begin{remark}
    For a finite field $k$, consider the function \footnote{not bothering with any set-theoretic issues} 
    \begin{align*}
        \psi: \mathcal{V}_k & \to   \mathbb{Z}\\
        X & \mapsto |X(k)|
    \end{align*}
    \begin{enumerate}
        \item If $X$ and $Y$ are isomorphic $k$-varieties, then $\psi(X) = \psi(Y)$
        \item If $Y \subset X$ is closed subvariety, then $\psi(X \setminus Y) = \psi(X) - \psi(Y)$ 
              (This essentialy gives the equation $Z(X \setminus Y, t) = \frac{Z(X, t)}{Z(Y,t)}$)
        \item If $X$ and $Y$ are two varieties, then $\psi(X \times Y) = \psi(X)\psi(Y)$
              (This gives, for example, the identity $Z(X \times \mathbb{A}^n_k, t) = Z(X, q^nt)$ )
    \end{enumerate}
\end{remark}


This motivates the construction of the Grothendieck ring of varieties as the universal ring over which any map with these properties factors as a 
ring homomorphism. Then our generalized measures will simply be ring homomorphisms out of this universal ring.

\begin{definition}
    Let $k$ be a perfect field.\footnote{The definition can be altered to make sense even if $k$ is not perfect, 
    in this case we just have to account for the fact that the product of two varieties need not be reduced. 
    But since all our applications are either over finite fields or in characteristic zero this slight loss of generality will not affect us.}
    Consider the abelian group of formal linear combinations of isomorphism classes of varieties, subject to relations of the form
    \[
        [X \setminus Y] = [X] - [Y]
    \] where Y is closed in X.
    With multiplication given by
    \[
        [X][Y] = [X \times Y].
    \]
    This forms a ring, called the \emph{Grothendieck ring of varieties} and denoted by $\gring$. 
    
    A \emph{motivic measure} with values in a ring $A$ is a ring homomorphism $\mu \colon \gring \to A$. The identity function
    $id \colon \gring \to \gring$ is called the \emph{universal motivic measure}.
\end{definition}

Since the product is commutative up to isomorphism this is a commutative ring with 1 (equal to $[\spec(k)]$). 
The cut and paste relation furthermore gives us
\[
    0 = [\emptyset] - [\emptyset] = [\emptyset \setminus \emptyset] = [\emptyset]
\]


\begin{example}
    \label{projSum}
    Using the decomposition $\mathbb{P}_k^n = \mathbb{P}_k^{n-1} \coprod \mathbb{A}_k^n$ where $\mathbb{P}_k^{n-1}$ is a (closed) hyperplane
    in $\mathbb{P}_k^n$  we get $[\mathbb{P}_k^n] = [\mathbb{P}_k^{n-1}] + [\mathbb{A}_k^n]$ in the Grothendieck ring.
    Inductively this yields the identity 
    \[
        [\mathbb{P}_k^n] = \sum_{m=0}^n [\mathbb{A}_k^1]^m
    \]
    %XXX: more natural place for introducing this convention?
    We also denote the isomorphism class of the affine line as $\mathbb{L}$.
\end{example}

%XXX: move?
\begin{remark}
    By \cite[Prop. 10.1 (d)]{Ha} the product of two smooth varieties over $k$ is again smooth. Hence the isomorphism classes of smooth
    irreducible complete varieties form a multiplicative monoid, in the following denoted by $\mathcal{M}$.
\end{remark}

We now investigate the structure of the Grothendieck ring. First we note that it suffices to take irreducible varieties to generate $\gring[k]$
To see this, take a variety $X$ with irreducible componets $X_1, \cdots, X_k$ and set ${U_i \coloneqq X_i \setminus \bigcup_{i \neq j} X_j}$ and
$U \coloneqq \bigcup_i U_i$. By construction, the last union is actually a disjoint union, and hence $[U] = \sum_i [U_i]$. Together this gives
\[
    [X] = [X \setminus U] + \sum_i [U_i].
\] Now the $U_i$ are irreducible as open subsets of irreducible sets, and $[X \setminus U]$ has dimension
strictly smaller than the dimension of $X$ since we removed a proper open subset from each irreducible component. Hence we can inductively
rewrite $[X \setminus U]$, loosing at least one dimension in each step, with the base case of $\dim  X = 0$ being trivial, as $X$ is then
just a collection of points.

In characteristic zero we can use a weak form of Hironaka's Theorem on the resolution of singularites, namely that for every irreducible, 
projective variety $X$ there is a smooth, projective variety that is birational to $X$ to restrict the set of needed generators even more. 
This observation culminates in a structure theorem about the Grothendieck ring proven by Bittner in \cite{Bittner}
%TODO: cite hironaka


\begin{theorem}[Bittner, {\cite[Thm. 3.1]{Bittner}}]
    \label{bittner}
    Let $k$ be a field of charakteristic zero. Then $\gring[k]$ is generated by smooth, integral, projective varieties.
    Furthermore, if $k$ is algebraically closed, it suffices to consider relations of the form
    \[
        [X] - [f^{-1}(Z)] = [Y] - [Z]
    \]
    where $X$,$Y$ are smooth complete varieties and $f \colon X \to Y$ is a morphism which is a blowup with a smooth center $Z \subset Y$.     
\end{theorem}
\begin{proof}
    We will only proof the first part of the statement.
    As remarked before, we can restrict our attention to irreducible varietes.  We once again argue by induction on the dimension.
    So let $X$ be a irreducible variety of dimension $n$, by choosing some nonempty affine subvariety and passing to its projective closure we
    find $X'$ projective, irreducible and birational to $X$. Resolving singularities we find $\widetilde{X}$ smooth, projective, irreducible
    birational to $X$. Hence we find isomorphic open subset $U \subset X$, $V \subset \widetilde{X}$ hand hence
    \[
        [X \setminus U ] - [\widetilde{X} \setminus V] = [X] - [U] - [\widetilde{X}] + [V] = [X] - [\widetilde{X}].
    \]
    Thus $[X]$ can be written as the sum of $[\widetilde{X}]$ and some varieties of lower dimension for which we can invoke the induction
    hypothesis.
\end{proof}

We will use the slightly stronger conclusion we just shown, hence we state it as a separate corollary.

\begin{corollary}
    \label{decomp}
    For each (irreducible) variety $X$ there exists a smooth, irreducible, projective variety $Y$ birational
    to $X$ such that $[X] = [Y] + \sum_i^N m_i[W_i]$ where the $[W_i]$ are smooth, irreducible, projective varieties of dimension strictly
    smaller than the dimension of $X$.
\end{corollary}



%XXX: More examples?
%TODO: symmetric product
%TODO: Example: symmetric product of P_k^1
%TODO: Motivic zeta function
%TODO: Examples for zeta functions (at least \zeta(P^1)
%XXX: keep the _k subscripts when writing A_k^n and P_k^n?

\section{Symmetric Products and the Motivic Zeta Function}

We now want to make the concept of the symetric product precise, i.e. how it can be defined as the quotient of the (regular) product by the
action of the symmetric group.

\begin{definition}
    Let $X$ be a $S$-Scheme, $G$ a group acting on $X$ via $S$-automorphisms (i.e there is a group homomorphism $\sigma \colon G \to Aut_S(X)$,
    we also write $\sigma_g$ for the image of $g$ under $\sigma$.) 
    Then a variety $Y$ together with a morphism $\pi \colon X \to Y$
    over $S$ is called the quotient (also denoted by $\faktor{X}{G}$) of $Y$ by $G$ if $\pi$ is $G$-invariant for all $g \in G$ 
    $\pi \circ g = \pi$ and is universal in this respect, i.e. for every other $G$-invariant morphism $f \colon X \to Z$ there exists a unique
    $G$-invariant morphism $g \colon Z \to Y$that makes the following diagram commute
    \begin{equation*}
        \begin{tikzcd}
            X \arrow{r}{f} \arrow{d}{\pi} & X/G \\
            Z \arrow[dashrightarrow]{ur}{g}
        \end{tikzcd}
    \end{equation*}
\end{definition}

We first construct the quotient in the affine case. Assume $X = Spec(A)$ with $G$ acting on $X$ (from the right) via automorphisms.
By the equivalence of categories between affine schemes and commutative rings, we get a left-action of $G$ on $A$, and we simply get the quotient
as the affine scheme $X \to X/G = Spec(A^G)$ coming from the subring of $G$-invariant elements, together with the natural inclusion $A^G \to A$.
By construction this is the categorial quotient defined above.

We now turn to a relevant special case: the quotient of an quasiprojective variety $X$ by the action of a finite group $G$, using the following
abstract result from \cite{SGA1}

\begin{lemma}[{\cite{SGA1}[Prop. 1.8]}]    
    Let $X$ be a scheme with a finite group $G$ acting on it via automorphisms. If there exists an open cover by $G$-invariant affine sets
    then the quotient $X/G$ exists.
    We call such an action \emph{admissible}
\end{lemma}

\begin{corollary}
    If $X$ is quasiprojective, then the quotient by the action of a finite group always exists.
\end{corollary}
\begin{proof}
    Let $X \subset \mathbb{P}_k^n$ be locally closed with a finite group $G = \{g_1,\cdots,g_m\}$ acting on it. 
    Consider the orbit $\{g_1(x),\cdots,g_m(x)\} \subset X$ of a point $x \in X$, and denote by $\mathfrak{p}_1,\cdots,\mathfrak{p}_m$ the 
    corresponding graded prime ideals corresponding to $\bar{\{x\}}$, and by $\mathfrak{q}$ the graded prime ideal corresponding to
    $\bar{X} \setminus X$ (taken to be the irrelevant ideal if this set is empty). 
    Hence the ideal $\mathfrak{q}$ is not contained in any of the $\mathfrak{p}_i$s hence, by the graded version of the prime
    avoidance lemma we find an element with positive degree $y \in \mathfrak{q}$ that is contained in none of the $\mathfrak{p}_i$ and hence
    the hypersurface $V_+(y)$ contains $\bar{X} \setminus X$ but not the $g_1(x),\cdots,g_m(x)$, so $U \coloneqq D_+(y)$ 
    is an affine set that lies in $X$ and contains the orbit of $x$ under $G$. Now consider $U' \coloneq \cap_{g \in G} g(U)$. This is clearly
    $G$-invariant, and since $U$ contained the orbit of $y$ we still find that $y$ lies in $U'$. Hence we have found a open cover by
    $G$-invariant affine sets, and thus the quotient exists by the above lemma.
\end{proof}

\begin{remark}
    If $X = \Proj(S)$ and the $G$-action is a \emph{graded} action (i.e. the induced action on $S$ respects the grading, we may construct
    the quotient, like in the affine case, as $X/G = \Proj(S^G)$
\end{remark}

\begin{definition}
    Let $X$ be a quasiprojective variety, then the $n$-fold \emph{symmetric product} $\Sym^n(X)$ is defined as the quotient of $X^n$
    (which is again quasiprojective via the Segre-Embedding) by the natural action of the symmetric group in $n$ letters.
\end{definition}

\begin{example}
    
\end{example}

\begin{definition}
    Let $\mu \colon \gring \to A$ be a motivic measure and $X$ a variety. Then we define the \emph{motivic zeta function} 
    $\zeta_{\mu}(X,t) \in A[[t]]$ as
    \[
        \zeta_{\mu}(X,t) \coloneqq \sum_{n=0}^\infty \mu([\Sym^n(X)])t^n
    \]
\end{definition}

\begin{example}
    We can now compute a first zeta function, namely $\zeta_{id}(\mathbb{P}_k^1, t)$. Using the identity and notation from example \ref{projSum}
    we calculate
    \begin{align*}
        \zeta_{id}(\mathbb{P}_k^1, t) & = \sum_{n=0}^\infty [\Sym^n({\mathbb{P}_k^1})]t^n 
                                       = \sum_{n=0}^\infty \big(\sum_{k=0}^n \mathbb{L}^k\big) t^n \\
                                       & = \sum_{n=0}^\infty \sum_{k=0}^n (\mathbb{L}t)^k t^{n-k} \\
                                      & = \Big( \sum_{n=0}^\infty t^n \Big) \Big( \sum_{n=0}^\infty (\mathbb{L}t)^n \Big) 
                                        = \frac{1}{(1-t)(1-\mathbb{L}t)} %TODO: add a fullstop that does not look completely shitty
    \end{align*}
    Hence $\zeta_{id}(\mathbb{P}_k^1, t)$ is in fact a rational function.
\end{example}

Kapranov proves in \cite{kapranov} that the zeta function of a curve with coefficents in a field is in fact always rational. 
In their paper \cite{MR1996804} Larsen and Lunts prove that in the case of surfaces this is false in general.
Later the same authors gave a more precise characterisation when the zeta function of a complex surface is rational 
(namely if and only if the Kodaira dimension is $-\infty$. See \cite{LL2}).

We will proof the following statement from \cite{MR1996804}
\begin{theorem}
    \label{irrational}
    Assume that $k = \mathbb{C}$. There exists a field $\mathcal{H}$ and a motivic measure $\mu \colon \gring \to \mathcal{H}$ with the following
    property: if $X$ is a smooth complex projective surface such that $P_g(X)=h^{2,0}(X) \ge 2$, then the zeta-function $\zeta_{\mu}(X,t)$
    is not rational.
\end{theorem}

\section{Constructing Motivic Measures}

We will now construct a measure $\mu_h$ for which we will show irrationality of the zeta function.

The first important result on the way is a lemma that helps us to construct such measures by extending maps from the monoid of smooth, irreducible
and complete varieties. For this we quickly recall the notion of a monoid ring.

\begin{definition}
    Let $G$ be a (multiplicative) monoid. Write $\mathbb{Z}[G]$ for the free abelian group over $G$, i.e. the set of formal sums
    $\sum_{g \in G} n_g g$ with the $m_g$ being integers, only finitely many non zero.
    Defining multiplication as
    \[
        (\sum_{g \in G} n_g g)(\sum_{g \in G} m_g g) = (\sum_{g \in G} (\sum_{hl = g} n_h m_l) g)
    \]
    this forms a ring: the \emph{monoid ring} over $G$.
\end{definition}

\begin{theorem}
    \label{th1}
    Set $k = \mathbb{C}$. Let $G$ be an abelian monoid and $\mathbb{Z}[G]$ the corresponding monoid ring. As above, denote
    by $\mathcal{M}$ the multiplicative monoid of irreducible smooth complete varieties. Let
    \[
        \psi: \mathcal{M} \to G
    \]
    be a homomorphism of monoids such that
    \begin{enumerate}
        \item $\psi([X]) = \psi([Y])$ if $X$ and $Y$ are birational;
        \item $\psi([\mathbb{P}^n]) = 1$ for all $n \ge 0$.
    \end{enumerate}
    Then $\psi$ can be uniquely extended to a ring homomorphism 
    \[
        \phi: \gring[\mathbb{C}] \to \mathbb{Z} [G]
    \]
\end{theorem}

%XXX: maybe rewrite the proof in a more "hands on" fashion to make clear why we check what we check and how \phi is defined
%XXX: why is \phi multiplicative?
\begin{proof}
    By \ref{bittner} we know that the Grothendieck ring (as an abelian group) is generated by $\mathcal{M}$, so after extending $\psi$ to
    the free abelian group over $\mathcal{M}$ 
    we have to check that $\psi$ preserves the relations of the blowup presentation of $\gring[\mathbb{C}]$ and hence factors over it
    , i.e.\ given a blow-up $f \colon X \to Y$ with smooth center $Z \in X$ then $\psi([X]) - \psi([f^{-1}(Z)]) = \psi([Y]) - \psi([Z])$  
    Note that $[X]$ and $[Y]$ are birational since $f$ is a blowup.
    
    Now since $X$ and $Z$ are nonsingular the exceptional divisor $f^{-1}(Z)$ is isomorphic to 
    the projective space bundle  $\mathbb{P}(\mathcal{I}/\mathcal{I}^2)$ 
    where $\mathcal{I}$ is the ideal sheaf associated to $Z$ (see for example \cite[II.8, Thm 8.24]{Ha}),
    which is a $\mathbb{P}^n$-bundle, i.e.\ locally trivializes to $\mathbb{P}_k^n \times U$ for some open set $U \subset Z$
    and hence is birational to $\mathbb{P}_k^n \times Z$. 
    We can thus write $\psi([f^{-1}(Z)])$ as  
    \[
        \psi([f^{-1}(Z)]) = \psi([Z \times \mathbb{P}^n]) = \psi([Z][\mathbb{P}^n]) = \psi([Z])\psi([\mathbb{P}^n) = \psi([Z])
    \]
    Hence we can linearly extend $\psi$ to define a morphism $\phi \colon \gring[\mathbb{C}] \to \mathbb{Z} [G]$
\end{proof}

Before constructing a suitable monoid homomorphism that will extend to our desired measure, we recall the definition of the Hodge numbers
of a variety as well as two basic results (all these are excercises in \cite[Ch. II]{Ha}).

\begin{definition}
    Let $X$ be a projective variety. The $i$-th \emph{Hodge number} $h^{i,0}$ is given as the dimension of $H^0(X,\Omega^i_X)$, 
    which is a finite dimensional $k$-vector space by a theorem of Serre (\cite[§3 Prop. 7]{FAC}).
    %XXX: why not directly define for 
    We will also write $P_g(X)$ for $h^{d,0}(X)$, called the geometric genus of a nonsingular variety $X$, with $d$ being the dimension of $X$.
\end{definition}

\begin{lemma}[{\cite[II Ex. 8.8]{Ha}}]
    \label{bir}
    The Hodge numbers $h^{i,0}$ are a birational invariant of a variety: let $X, Y$ be birationally equivalent, smooth, projective varieties
    then $h^{i,0}(X) = h^{i,0}(Y)$.
\end{lemma}

\begin{lemma}[Chow, {\cite[II Ex. 4.10]{Ha}}]
    Every proper variety is birational to a projective variety.
\end{lemma}

\begin{definition}
    Denote by $C \subset \mathbb{Z}[t]$ the multiplicative monoid of polynomials with positive leading coefficient.
    For a smooth projective complex variety $Z$ of dimension $d$ define
    \[
        \Psi_h(X) := 1 + h^{1,0}(X)t + \cdots + h^{d,0}(X)t^d \in C.
    \]
    (Note that the Hodge numbers might be zero, but $\Psi_h(X)$ still has a positive leading coefficient, 
    just its degree might be smaller than the dimension of $Z$)
    By Chow's lemma we can also define $\Psi_h$ for a smooth complete variety $Z$ by choosing a smooth projective variety $X$ which is birational
    to $Z$ and setting $\Psi_h(Z) := \Psi_h(X)$. This is well defined by Lemma \ref{bir}.
\end{definition}

We now check that $\Psi_h$ satisfies all conditions of Theorem \ref{th1}. Independence of birational equivalence class was the content of Lemma
\ref{bir}. 

To check multiplicativity, we use the following Lemma.

\begin{lemma}
    Let $X,Y$ be smooth, irreducible, projective $k$-varieties. Then the following equality holds:
    \[
        h^{p,0}(X \times_k Y) = \sum_{i+j=p} h^{i,0}(X)h^{j,0}(Y)
    \]
\end{lemma}
\begin{proof} TODO \end{proof}

With this we directly calculate
\begin{align*}
    \Psi_h(X \times_k Y) & = \sum_p h^{p,0}(X \times_k Y) = \sum_p \sum_{i+j=p} h^{i,0}(X)h^{j,0}(Y) \\
    & = \Big (\sum_p h^{p,0}(X) \Big) \Big(\sum_p h^{p,0}(Y)\Big) = \Psi_h(X) \Psi_h(Y)
\end{align*}

%TODO: Check \Psi_h(P_k^n) = 1 (use Euler sequence)

\section{Irrationality}

The constructed motivic measure does not yet yield values in a field, so we would like to pass to the quotient field of $\mathbb{Z}[\mathcal{M}]$.
That we are able to do so is the content of the next lemma.

\begin{lemma}
    Let $A$ be a factorial ring, and $S \subset A$ a multaplicative submonoid such that $1$ is the only unit in $S$. Then the monoid ring
    $\mathbb{Z}[S]$ is a polynomial ring (in possibly infinitely many variables), and hence an integral domain.
\end{lemma}
\begin{proof}
    Since $A$ is factorial, every $s \in S$ has a unique factorization, and since 1 is the only unit in $S$, 
    $s$ can be uniquely written as product of prime elements, hence if we take $B$ to be the polynomial ring over the formal
    variables $\{x_s | s \in S, s\ \text{prime}\}$ we get an isomorphism of rings
    \begin{align*}
        B &\to \mathbb{Z}[S] \\
        x_s &\mapsto s
    \end{align*}
\end{proof}

\begin{definition}
    Denote by $\mathcal{H}$ the quotient field of $\mathbb{Z}[C]$ where, as above, $C$ denotes the submonoid of $\mathbb{Z}[t]$ 
    consisting of polynomials with positive leading coefficients. Since $-1$ is not contained in $C$ it satisfies the conditions of the
    previous lemma
    We define the motivic measure $\mu_h \colon \gring \to \mathcal{H}$ as the measure obtained extending $\Psi_h$ as by Theorem \ref{th1}
\end{definition}

\begin{lemma}
    \label{same}
    Let $Y_1,\cdots,Y_s,Z$ be irreducible varieties of dimension d over a field of characteristic zero 
    such that $\mu_h([Z]) = \sum_i n_i \mu_h([Y_i])$ for some $n_i \in \mathbb{Z}$ and $P_g(Z) \neq 0$ then $P_g(Z) = P_g(Y_i)$ for some $i$
\end{lemma}
\begin{proof}
    By Corollary \ref{decomp}, we find smooth, irreducible projective varieties $\overline{Z},\overline{Y_1},\cdots, \overline{Y_s}$ in the same
    birationial class as the original varieties allowing us to rewrite the original equality as 
    \[
        \mu_h([\overline{Z}]) = \sum n_i\mu_h([\overline{Y}_i]) + \sum l_i \mu_h([X_i])
    \]
    the $X_i$ being smooth irreducible varieties of dimension $<d$.
    But now, since $\mu_h$ was obtained as an extension of $\Psi_h$ which was defined on smooth, irreducible, projective varieties, this is
    actually an equation in $\mathbb{Z}[C]$, namely
    \[
        \Psi_h(\overline{Z}) = \sum n_i \Psi_h(\overline{Y}_i) + \sum l_i\Psi_h(X_i)
    \]
    Since there are no (additive) relations between elements of $C$ in the monoid ring, one of the polynomials $\Psi_h(\dots)$ on the 
    left hand side, must actually be the polynomial on the left hand side, but with the dimension of 
    $X_i$ being strictly smaller than that of $Z$ and the assumption that
    $P_g(\overline{Z}) = h^{d,0}(\overline{Z}) \neq 0$ it cannot be one of the $\Psi_h(X_i)$ because all these polynomials have strictly smaller
    degree. Hence we have $\Psi_h(\overline{Z}) = \Psi_h(\overline{Y}_i)$ for some $i$ and in particular, since they are of the same dimension, the genus of
    $\overline{Z}$ and $Y_i$ must agree.
\end{proof}

Now let $X$ be a smooth projective surface with $P_g(X) \geq 2$. We will show that $\zeta_{\mu_h}(X,t) \in \mathcal{H}[[t]]$ is not rational,
thus proving Theorem \ref{irrational}. We need one technical lemma about the genus of symmetric powers of X, which we cannot proof here.

\begin{lemma}[{\cite[Lem. 3.8]{MR1996804}}]
    \label{genus}
    Let $X$ be a smooth, projective surface over $\mathbb{C}$. Then
    \[
        P_g(\Sym^n(X)) = \binom{P_g(X) + n - 1}{P_g(X) - 1}
    \]
\end{lemma}

We will use the following rationality criterion for power series.

\begin{lemma}
    Let $k$ be a field. Then a power series $\sum a_it^i \in k[[t]]$  is a rational function (i.e. an element in $k(t)$) if and only if
    there exist $n, n_0 > 0$ such that for each $m > n_0$ the determinant
    \[
        \begin{vmatrix}
            a_m     & a_{m+1} & \dots & a_{m+n} \\
            a_{m+1} & a_{m+2} & \dots & a_{m+n+1} \\
            \vdots  & \vdots  & \ddots & \vdots \\
            a_{m+n} & a_{m+n+1} & \dots & a_{m+2n}
        \end{vmatrix}
    \]
    vanishes.
\end{lemma}
\begin{proof}
    See \cite[Section 5.3, Lem. 1]{MR0241425}
\end{proof}

\begin{proof}[Proof of Theorem \ref{irrational}]
    Assume that $\zeta_{\mu_h}(X,t)$ is rational, hence by the above criterion there is a $n$ such that for $m$ big enough
    \begin{equation}
        \begin{split}
            \sum_{\sigma \in S_{n+1}} \mu_h ( \prod_{i=1}^{n+1} \Sym^{m + i + \sigma(i) - 2}(X)) = 0  \\
        \Leftrightarrow
        \mu_h \Big ( \prod_{i=0}^n \Sym^{m + 2i}(X) \Big ) = - \sum_{\substack{\sigma \in S_{n+1} \\ \sigma \neq id}} 
        \mu_h \Big ( \prod_{i=0}^n \Sym^{m + i + \sigma(i+1) - 1}(X) \Big )
    \end{split}
    \end{equation}
    Now we can apply Lemma \ref{same} to conclude that there is a permutation $\sigma$, not the identity permutation, such that
    \[
        P_g \Big(\prod_{i=0}^n \Sym^{m+2i}(X)\Big) = P_g \Big( \prod_{i=0}^n \Sym^{m+i+\sigma(i+1) - 1}(X) \Big)
    \]
    By Lemma \ref{genus} (and using the fact that the genus is multiplicative), we get
    \[
        \prod_{i=0}^n \binom{P_g(X) + m + 2i - 1}{P_g(X) - 1} - \prod_{i=0}^n \binom{P_g(X) + m + i + \sigma(i+1) - 2}{P_g(X) - 1} = 0
    \]
    By assumption $P_g(X) \geq 2$ hence the left hand side, considered as an polynomial in $m$, is not the zero polynomial. So by taking
    $m$ large enough we obtain a contradiction.
\end{proof}




\bibliography{mybib}{}
\bibliographystyle{alpha}
\end{document}
